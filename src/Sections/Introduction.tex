\chapter{Introduction}

	With the total amount of data in the world growing steadily at a fast rate, 
	the need to analyze large amounts of data in datasets, or big data, is a major 
	key to productivity growth, and innovation [10]. With this increase in 
	information and the detail of the information available across the web, 
	the internet will continue to provide a basis for the continued growth of 
	data in the future. Data is a part of every industry and business in today’s 
	world, from retail to government. Big data and analysis can help to create
	 value in these industries by making information more usable, showing more 
	 accurate performance measures, and providing detailed analytics that 
	 lead to better decision making.

	In this data-driven time, it is important to be able to make decisions 
	quickly using the data that is available. Big data visualization is an 
	effective way to present the important information amongst the large amount 
	of data used and helps to drive complex analysis [11]. With the use of big 
	data analysis, it can make otherwise too large amounts of data meaningful 
	to those who looking to interpret significance from it.

	The Massachusetts High Technology Council, or MHTC, is a group of technological, 
	professional and higher education executives across the state of Massachusetts. 
	They are comprised of higher education CEOs, senior executives from Massachusetts 
	and more. For over thirty-eight years, MHTC has advocated for various policies 
	and programs to create and keep both a healthy and competitive business climate. 
	Their goal is to establish and keep Massachusetts as a competitive place for 
	successful business and talent building, including a specific focus within the 
	technology sectors. In part because of the MHTC and its members, who are 
	regarded as the region’s most venerable technology association, Massachusetts 
	is regarded as one of the top competitive areas for businesses that are “high tech” [12].

	MATTERS, the Massachusetts Technology, Talent, and Economic Reporting System, 
	was developed by MHTC along with Worcester Polytechnic Institute (WPI) and other 
	institutions as a collaborative effort designed to aid users in understanding a 
	variety of public data in an effort to help make Massachusetts the top location 
	for high technology businesses. MATTERS uses this data collected to both measure 
	and evaluate the current state of Massachusetts compared to other states in the 
	country and provides policy makers and advocates with easily accessible and searchable 
	dynamic data to assist in their efforts regarding decisions to help  retain and 
	grow business in the state. MATTERS is able to provide a way to centralize a wide 
	range of talent, cost, and economic metrics and state rankings from federal and 
	state government sources, non-profit organizations and media outlets into one 
	location for its users [13]. 

	Previously to our project, WPI IQP, MQP and graduate student teams worked to 
	develop the MATTERS dashboard as it was when we began working. The goal of our 
	Major Qualifying Project (MQP) is to provide additional features to the MATTERS 
	website that will give our users additional ways to interact with the data and 
	metrics found within the site. The idea for the metric builder was presented by 
	the MHTC members as a means for the MHTC to be able to visualize their own 
	indicator that they were in the process of creating. Users will be able to 
	create an account on the MATTERS site and be able to build and display the 
	results of their own unique metric formulas for ranking states. Users will 
	also be able to retrieve any of the data points found throughout the site 
	for their own personal use through the creation of our own API. These new 
	features will provide authorized users an opportunity to use and share the 
	data in even more innovative ways, aiding in the MHTC’s goal toward making 
	Massachusetts a leading state for high technology business. These two features 
	create further access to the site’s data and additional ways to manipulate and 
	visualize the data to interpret it in new ways. These features also create an 
	expanded user base for MATTERS and different privileges for different types of 
	users, such as the already existing administrative users, and general users who 
	also would like to create their own indicators.