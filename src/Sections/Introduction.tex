\chapter{Introduction}

	With the total amount of data in the world growing steadily at a fast rate, 
	the need to analyze large amounts of data in datasets, also called big data, is a 
	key driver for productivity growth and innovation \cite{bigdata}. Examples of data sets found throughout society include social media accounts, census data, medical records, and ticket sales \cite{bdex}. With this increase in 
	information often freely available across the web, 
	our society will continue to provide a basis for the continued growth of 
	data in the future. Data is a part of every industry and business in today's 
	world from retail to government. Big data can help to create
	 value in these industries by making information more usable, showing more 
	 accurate performance measures, and providing detailed analytics on this data that 
	 leads to better decision making.

	In this data-driven time, it is important to be able to make decisions 
	quickly using the data that is available. Big data visualization is an 
	effective way to present the important information amongst the large amount 
	of data used and helps to drive complex analysis \cite{bigdata}. Big 
	data analytics tools can make otherwise too large amounts of data meaningful 
	to those who looking to interpret significance from it.

	The Massachusetts High Technology Council, or MHTC \cite{mhtc}, is a group of technological, 
	professional and higher education executives across the state of Massachusetts. 
	They are comprised of higher education CEOs, senior executives from Massachusetts 
	and more. For over thirty-eight years, MHTC has advocated for policies 
	and programs to create and keep both a healthy and competitive business climate in Massachusetts. 
	Their goal is to establish and keep Massachusetts as a competitive place for 
	successful business and talent building, including in particular with a specific focus on the 
	technology sectors. In part because of the MHTC and its members, who are 
	regarded as the region's most venerable technology association, Massachusetts 
	is regarded as one of the top competitive areas for businesses that are “high tech” \cite{mhtc}.

	The Massachusetts Technology, Talent, and Economic Reporting System,also known as MATTERS, 
	was developed by MHTC along with Worcester Polytechnic Institute (WPI) and other 
	institutions as a collaborative effort designed to aid users in understanding a 
	variety of public data in an effort to help make Massachusetts the top location 
	for high technology businesses. MATTERS uses this data collected to both measure 
	and evaluate the current state of Massachusetts compared to other states in the 
	country. It provides policy makers and advocates with easily accessible and search-able 
	dynamic data to assist in their efforts regarding decisions to help  retain and 
	grow business in the state. MATTERS integrates a wide 
	range of talent, cost, and economic metrics and state rankings from federal and 
	state government sources, non-profit organizations and media outlets into one 
	location via a data warehouse for its users \cite{about}. 

	In this effort, Worcester Polytechnic Institute, is the technology partner of this collaboration, and the WPI undergraduate and graduate students have worked sicne 2014 on the development of this MATTERS dashboard \cite{wpi}. The full team can be found at http://davis.wpi.edu/dsrg/PROJECTS/MATTERS/Members.html. Previously to our project, WPI IQP, MQP and graduate student teams worked to 
	develop the MATTERS dashboard as it was when we began working. These previous projects are discussed 
	in the \textit{Background and Related Work Section}. The goal of our 
	Major Qualifying Project (MQP) is to provide additional features to the MATTERS 
	website that will give our users advanced features to interact with the data and 
	metrics found within the site. These metrics are measures of quantitative assessments which allow people to track performance and make comparisons on the specific data values \cite{metrics}. One key feature for our MQP team is to develop the Metric Builder 
	as a means for the MHTC to develop and then work with their own 
	\textit{MATTERS} indicators, which are statistics used in order to measure the current conditions of different sectors as well as examine trends and predict future trends of the data, that they have been in the process of creating \cite{indicator}. In particular, MATTERS users will be able to 
	create an account on the MATTERS site, and proceed to design the their own unique indicator by developing a metric formula for ranking states from the preexisting metric data. As a second core feature, targeted by our MQP, we will provide a programmatic interface to the MATTERS site. The aim is that this way users will 
	also be able to retrieve any of the data points found throughout the site 
	for their own personal use via a standard API, which is "a set of requirements that govern how one application can talk to another" \cite{apidef}. The API is an important tool to add convenience for developers who wish to access the data and can save users a lot of time \cite{apidef}. In total, these new 
	features will provide authorized users an opportunity to use and share the 
	data in even more innovative ways, aiding in the MHTC's goal toward making 
	Massachusetts a leading state for high technology business. These two features 
	provide further access to our site's data and additional ways to manipulate and 
	visualize the data to interpret it in new ways. These features also create an 
	expanded user base for MATTERS. Different privileges are offered to different types of 
	users, such as the already existing administrative users, as well as general users who 
	also would like to create their own indicators.