\chapter{Methodology}

\section{Design Choices}

\subsection{Metric Builder}

Implementing a Metric Builder, the core question was how to store and compute user metric. We have come to these two options
\begin{enumerate}
  \item
    Store user metric data in databse
    \begin{itemize}
      \item
        Pros
        \begin{itemize}
          \item
            Fast to retrive; n computation needed
        \end{itemize}
      \item
        Cons
        \begin{itemize}
          \item
            Waste of space it database
          \item
            User metric data remains ststic; update in underlying data does not cause an update in the user metric
        \end{itemize}
    \end{itemize}
  \item
    Store user metric as a metadata only and compute values on-the-fly (on-demand)
    \begin{itemize}
      \item
        Pros
        \begin{itemize}
          \item
            Efficient use of database space
          \item
            User metric data is always in sync with base data
        \end{itemize}
      \item
        Cons
        \begin{itemize}
          \item
            Might be time costy to compute
        \end{itemize}
    \end{itemize}
\end{enumerate}

Looking at the cons of each option, we attempted to approximate how bad they are. Having $n$ users each creating $m$ user metrics each of which has data for $k$ years we have $m \cdot n \cdot k \cdot 50$ entries in database. This data would be a snapshot. If any data point in metrics changes, user metrics remain out of sync. \\

For the second option, it was diffucult to approximate time it takes to compute a user metric as it depends on server load and number of metrics involved in the user metric. We implemented the second option and benchmarked it wth 3 users trying to compute largest possible user metric (all metrics with all states selected). It took around $1.5$ seconds to load all $3$ user metrics. This drove our design decision towards the second option - computing metrics on-the-fly.

\subsection{API}

\section{Changes to the system and database}
