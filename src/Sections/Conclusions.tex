\chapter{Conclusion}

	In this project we have built two services for the MATTERS site. The API 
	provides a way for other computer-based systems to work with the MATTERS 
	system. Specifically, it is now possible to query MATTERS data directly, 
	bypassing the presentation layer. The other feature is a Metric Builder. 
	This subsystem allows users to create their own unique metrics based on 
	the existing metrics. Users are able to work with their custom metrics seamlessly 
	in the Data Explorer as if they were just regular ones.  

	\section{Recommendations}
		
		Recommendations for other teams, either MQP or other teams, working on improving the MATTERS system in the future are as follows:
		
		\begin{itemize}
			\item MATTERS system
				\begin{enumerate}
					\item
						Let users define more complex equations for their user metrics.
						User should not be limited by simple weighted average formula.
					\item
						Let users suggest their own data or data sources. Right now 
						MATTERS' administrators and data management team are working on
						adding data. By involving users in this process MATTERS may have 
						more complete, accurate and up-to-date data.
					\item
						Role management. Right now the system supports regular users and API users. 
						Each user metric must have one and only one author. It might be a good idea 
						to merge these two user entities and introduce shared user metrics that are 
						not binded to specific users.
				\end{enumerate}
			\item Software development
				\begin{enumerate}
					\item
						Refactor the client side and server side code. Since the system 
						has been developed by a number of teams with different design 
						patterns and approaches, there is a significant technical debt. 
						It might be worth spending some time rewriting some parts of the system 
						according to the latest coding and technical standards.
					\item
						Documentation. Right now there is a steep learning curve for the new 
						developers who start working on the project. Good documentation may 
						dramatically decrease the time it takes for developers to start working 
						on the project.
				\end{enumerate}
		\end{itemize}
