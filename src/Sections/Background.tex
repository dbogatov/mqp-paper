\chapter{Background and Related Work}

	The creation of a Metric Builder page and an API for the MATTERS site must leverage as well as 
	seamlessly integrate with the overall MATTERS site, which has been created by 
	several prior project groups at WPI. These students worked to establish the initial features 
	and the design framework for MATTERS to exist. The initial development of MATTERS in Spring of 2014 
	started with a team of eleven students composed of one undergraduate team completing their Interactive Qualifying Project (IQP), 
	and 7 WPI graduate students. This team of IQP students researched 
	and came to the first core decisions regarding the front-end visualization tools for 
	the MATTERS dashboard \cite{prevreport}. This includes the inclusion of decisions 
	regarding colors, and visualization types such as charts and tables, that would 
	be used to the represent the data within the system in a manner that was both 
	intuitive for the users and a quick way to show analytics results. More importantly, the IQP 
	team conducted a survey of the MHTC board members to determine their familiarity and 
	preferences concerning visual interaction components of dashboards. In parallel, 
	the graduate student teams developed the first back-end version for MATTERS. 
	They developed components to extract the desired data from diverse websites, 
	parse and clean this retrieved data, before then uploading it into a 
	data warehouse. This data would then be called upon to be shown in the various visual 
	methods for the users \cite{iqp}.

	Following this initial front-end and back-end design and implementation 
	of the MATTERS dashboard, a team of students working on their Major Qualifying 
	Project (MQP) at WPI improved the administration center and the 
	data integration for the dashboard. This involved the data integration Pipeline 
	Manager. In addition, the Administration Center's development to allow for MHTC 
	and other administrators who had access to be able to easily upload their 
	desired data and metrics, as well as an intuitive way to view this data on the 
	administrative end. This MQP team was able to integrate 20 new data sources 
	for the MATTERS dashboard, make the Administration Center usable for 
	non-technical users, and improve upon the look-and-feel of the MATTERS 
	dashboard as a whole \cite{iqp}.

	\section{System Architecture}
	
		The MATTERS Website consists of two subsystems - the data acquisition and management \emph{admin panel} and the visual-interaction \emph{dashboard}.
		While the \emph{admin panel} is used by administrators to work with data 
		(basic create, read, update, and delete, or CRUD, operations), in this project we focused on the \emph{dashboard} subsystem.
		This subsystem follows MVC (Model-View-Controller) architecture. You can see its visual 
		structure in Figure \ref{fig:sysarch}.
		
		\begin{figure}[t]
			\centering
			% Define block styles
\tikzstyle{request} = 
	[rectangle, dashed, rounded corners, draw, fill=gray!30, text width=7em, 
	text centered, minimum height=3em]
	
\tikzstyle{layer} =
	[rectangle, draw, fill=gray!50, 
	text width=10em, text centered, rounded corners, minimum height=4em]

\tikzstyle{line} = [draw]

\tikzstyle{explanation} = 
	[draw, ellipse,fill=gray!70, node distance=7cm,
	minimum height = 4em, text width=7em, text centered]
	
\begin{tikzpicture}[node distance = 2cm, auto]
	
	% Place nodes
	\node [layer] (view) {View. Client side};
	\node [explanation, right of=view] (viewexpl) {CSS, JS, HTML};
	
	\node [request, below left=0cm and 1cm of view] (ajax) {AJAX call};
	
	\node [layer, below of=view, node distance=3cm] (controller) {Controller. Server side};
	\node [explanation, right of=controller] (controllerexpl) {Java, Spring framework};
	
	\node [request, below left=0cm and 1cm of controller] (storedproc) {DB procedure};
	
	\node [layer, below of=controller, node distance=3cm] (model) {Model. Data layer};
	\node [explanation, right of=model] (modelexpl) {PostgreSQL};
	
	% Draw edges
	\path [line] (view) -- node {platform} (viewexpl);
	\path [line] (controller) -- node {platform} (controllerexpl);
	\path [line] (model) -- node {platform} (modelexpl);
	
	\path [line, -latex'] (view) -| node [near start] {request} (ajax);
	\path [line, latex'-] (ajax) |- node [near end] {response} (controller);
	
	\path [line, -latex'] (controller) -| node [near start] {request} (storedproc);
	\path [line, latex'-] (storedproc) |- node [near end] {response} (model);
	
\end{tikzpicture}
			\caption{MATTERS System architecture}
			\label{fig:sysarch}
		\end{figure}
		
		
		\subsection{Model}
			
			The MATTERSS' data layer is built with the PostgreSQL database. The main tables are:
			
			\begin{description}[itemsep=-1.5mm, align=right,labelwidth=3cm]
				\item [Statistics]
					holds metrics' data in the form of a year, state, metric and value tuple;
				\item [States]
					holds states' metadata like name, abbreviation and if it is a peer state;
				\item [Metrics]
					holds metrics' metadata like description, trend and category;							
			\end{description}
			
			The access to these tables is provided via stored procedures as a data access layer.
			
		\subsection{Controller}
		
			The business logic of the system resides on a Java-based server. Spring framework is 
			used as a server architecture solution. A front end request that comes to the system goes 
			through controllers which, in turn, invoke services. Services query the data layer and return 
			data stored	in models. These data models get pushed back to the front end.
			
		\subsection{View}
			
			The presentation layer is represented by the HTML based website. The client side generates 
			queries to the server via JavaScript AJAX calls, then use the returned data to generate the content.
			The website has a significant amount of logic on the client side as it needs to build 
			tables, plots and charts. The two main JavaScript libraries used on the client side are jQuery and D3.
			jQuery makes it easy to manipulate DOM while D3 is used for generating vector based plots and charts.
			
			Visual representation and displays are based on the responsive Twitter Bootstrap 3 framework, 
			which is a collection of CSS classes and small JavaScript snippets. 

	\section{Requirements}

		Before we could embark on creating an API for MATTERS as well as designing the Metric 
		Builder page for the site, we had to conduct research. This included to look into other 
		similar features of different sites and make decisions regarding implementation,
		design and security features for these aspects of the MATTERS site.

	\subsection{API and Documentation}

		An API is an "Application Programming Interface". APIs are tools that allow other 
		programs to interact with the software program that the API belongs to without 
		having to give someone complete access to your code base. In a sense they expose 
		some of the internal workings of software to the public but in a way that is 
		limited by what the API chooses to provide. APIs aid in providing a way to share 
		a potentially large amount of data in an efficient way. According to the chief 
		data officer of Philadelphia, Mark Headd, APIs "allow a specific audience to use 
		data more quickly, easily, and efficiently when they are looking to do something 
		specific with the information." APIs are beneficial for not only keeping certain 
		aspects of code private, but also for saving users time. Even in cases of open 
		source programs where all of the code for a program is visible, the
		code base can be so large that it is inefficient to search through the whole code base 
		for specific data \cite{govapi}. 

		APIs are beneficial from both a business perspective and for programs that to aim share 
		information and research with many people. An advantage of an API being an automated 
		tool is that it can process a large number of requests without any added cost and work 
		from the developers of the initial site. APIs also present the data requested in a manner that is useful 
		and easily manipulated for the users' purposes, especially if the users could be programmatic access. 
		For APIs whose goals are to share 
		certain data across systems there is a great benefit efficiency wise, as many manual 
		procedures are cut out by the APIs automatic generation. When these tasks are done 
		manually, they waste time and are both laborious and repetitive, as well as being more 
		prone to error; all of these things being costly on the developer end. While it is more 
		time consuming and costly to develop an API initially, the overall benefit once it is 
		developed is apparent and can be reaped long-term \cite{govapi}.

		Additionally, APIs not only encourage innovation and the ability to manipulate and 
		extrapolate data by external collaborators, they provide a means to do so that is 
		more secure for the group creating the API. They reduce the risk of how and what data 
		is obtains and encourage good practice for how to properly manage the data 
		available \cite{readwrite}.

	One important step to gaining the full benefit from an API is making sure that the API is 
	secure. This can be achieved through requiring authentication and authorization of users 
	before they gain access to the API's features. It is good practice to ensure that the users 
	are who they say they are and that they are given permission before using data or parts of 
	a developer's program. This also allows for a record of who is accessing the data, 
	in the event that there is an issue so that a specific user can be tracked down \cite{jisc}.

	In addition to creating the API for a site, it is necessary to create good documentation to 
	accompany the API, to make the API as user friendly and simplistic as possible. 
	API Documentation thus must be provided to the developers that details everything that they need to know in a concise 
	manner in order to use the API’s features with other applications. The API Documentation 
	needs to contain the classes, functions and other important aspects of the API \cite{cio}. 
	The documentation guide allows developers to easily interact with a site's API and their 
	own code, as many developers style their code in their own unique way. 

	\subsection{Metric Builder}

		The main service that is to be developed by our MQP team is the Metric Builder page. This service will be available 
		for the registered users on the MATTERS website only. A user will be able to select any 
		number of the metrics already available in the MATTERS database and combine their data using a custom formula to 
		create their own personal economic "indicator". The user can then use their created formula 
		to display the states' rankings based on this indicator. This will allow users to see and 
		evaluate states based on what they think is most important as a whole or a combination of 
		various factors, and interpret the data in their own way in a simple manner. By enabling users to 
		create their own data set from all the metrics available and then provide their own weights 
		to each of these data values, we are allowing users to see the competitive advantage that each state has 
		based on how important these metrics are for the user's purpose.

		It is important to ensure that different data types can be combined to create 
		one final index value. It is also important to consider the effect that certain data types will 
		have on the final value computed from the user's indicator formula. For example, the MATTERS database 
		contains rankings, percentages, nominal values, and other data types. Certain data types could 
		end up dominating in a user-created indicator when combined with significantly smaller value 
		data types. There are also complexities involved with certain data types that are represented in 
		the reverse of the majority of the data types. For example, both rankings and some other data 
		types use numerical values in which a lower value indicates a better performance by the state, 
		versus for the majority of other metrics the larger the value is the better a state is 
		performing in that area. This corresponds to the data type incompatibility problem, requiring us to learn about data normalization and transformation. 

		Finally, there are many metrics that do not contain a value for every single year that other 
		metrics hold a value, the missing value problem. Situations where one metric being used in the user's Metric Builder equation
		does not contain values for years that another metric being used in the same equation need to be 
		considered. There needs to be a uniform solution for what to do in these instances in a manner that 
		does not deceive the users of the indicator, as this would affect the calculations and subsequent 
		visualizations across years where these issues exist.

		These factors need to be taken into consideration to provide accurate and understandable 
		visualizations, when users create their own indicator in the Metric Builder. It is for 
		this reason good to consider special handling or removal of certain data types, as well as potential 
		ways to normalize the final numerical values received from the formulas the users have created in 
		the Metric Builder.

		Lastly, we researched various economic indicators in the literature and on other public sites to come up with an idea for how to allow our users to 
		combine sets of data to use for comparisons. One common approach in economic indicators is to use 
		weighted averages \cite{weightedaverage}. For example, one of the most popular economic indicators, 
		CPI, or the Consumer Price Index, uses weighted averages of data from hundreds of different consumer 
		goods and services to measure inflation and deflation. Each of the goods considered by the CPI is 
		weighted based on its importance \cite{cpi}.

		Weighted averages work by taking each number in a given data set, and multiplying the value by 
		its given weight. This will give you a new value reflecting the product of the data and its weight. 
		Lastly, we then add all of these products together from the data set to get the total value. Then, we add up the total 
		of all the weights. Finally divide the total value by the total weight in order to receive the weighted 
		average of your data set. The weights are used to show the importance that each specific item in a 
		data set has in order to calculate an overall value of all the data together \cite{economic}.
